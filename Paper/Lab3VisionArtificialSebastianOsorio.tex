\documentclass{llncs}
\usepackage[spanish]{babel}
\usepackage[utf8]{inputenc}
% \usepackage{amsmath}
% \usepackage{amssymb}
\usepackage{longtable}
\usepackage{graphicx}
\usepackage{float}
\usepackage{hyperref}
\usepackage{listings}
\usepackage{color}
\usepackage{subcaption} % Para el entorno subfigure

\usepackage{authblk}
\usepackage{listings}
%\usepackage[utf8]{inputenc}
\usepackage[T1]{fontenc}

%%%%%%%%%%%%%%%%%%%%%%%%%%%%%%%%%%%%%%%%%%%%%%%%%%%%%%%
%%%%%%%%%%%%%%%%%%%%%%%%%%%%%%%%%%%%%%%%%%%%%%%%%%%%%%%

\graphicspath{{./figures/}} % Ruta de las imágenes

\usepackage{listings}
\usepackage{xcolor}
\usepackage{multirow}
\usepackage{longtable}

\definecolor{codegreen}{rgb}{0,0.6,0}
\definecolor{codegray}{rgb}{0.5,0.5,0.5}
\definecolor{codepurple}{rgb}{0.58,0,0.82}
\definecolor{backcolour}{rgb}{0.95,0.95,0.92}

\lstdefinestyle{mystyle}{
    backgroundcolor=\color{backcolour},   
    commentstyle=\color{codegreen},
    keywordstyle=\color{magenta},
    numberstyle=\tiny\color{codegray},
    stringstyle=\color{codepurple},
    basicstyle=\ttfamily\footnotesize,
    breakatwhitespace=false,         
    breaklines=true,                 
    captionpos=b,                    
    keepspaces=true,                 
    numbers=left,                    
    numbersep=5pt,                  
    showspaces=false,                
    showstringspaces=false,
    showtabs=false,                  
    tabsize=2
}

\lstset{style=mystyle,language=Python}



%%%%%%%%%%%%%%%%%%%%%%%%%%%%%%%%%%%%%%%%%%%%%%%%%%%%%%%
\title{Laboratorio Nro 3: Análisis de la deforestación en la selva amazónica con técnicas de segmentación}

\author{\href{https://orcid.org/0000-0001-9379-4011}{Sebastián  Osorio}  $^{[1]}$}


\affil{Maestrante de Inteligencia Artificial}
\institute{[1] Universidad Internacional de la Rioja (UNIR), España}

\date {16 de junio de 2025}
\begin{document}
\maketitle

\section{Resumen}

\section{Introducción}
Existe un gran impacto medioambiental debido a la deforestación, especialmente en la selva amazónica. Este laboratorio se centra en el análisis de imágenes satelitales para detectar y segmentar áreas afectadas por la deforestación. Utilizando técnicas de segmentación de imágenes, se busca identificar patrones de cambio en la cobertura forestal a lo largo del tiempo.
El objetivo es desarrollar mediante técnicas de segmentación, un modelo capaz de detectar áreas deforestadas en imágenes satelitales de la selva amazónica. Además, se pretende calcular el area afectada por la deforestación y analizar la evolución de la cobertura forestal en diferentes periodos.
\section{Materiales y Métodos}
Se utilizó un conjunto de datos de imágenes satelitales de la selva amazónica, que incluye imágenes de diferentes años (2000-2019). Las imágenes fueron preprocesadas para mejorar la calidad y facilitar la segmentación. 


\section{Resultados}

\section{Conclusiones}

\bibliographystyle{IEEEtran}
\bibliography{bibliografia}


\end{document}